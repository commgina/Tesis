\begin{center}
\textbf{Resumen}
\end{center}

\vspace{2em}
Las vulnerabilidades definen la superficie de ataque de un sistema digital. Incluso en su configuración mínima, una aplicación expuesta a redes públicas o privadas incluye múltiples capas como firmware, sistema operativo, frameworks y aplicaciones con arquitecturas propias, todas susceptibles a vulnerabilidades conocidas y registradas en bases como \ACRshort{cve} \cite{CVE}. A pesar de contar con información pública sobre estas fallas, es común que se repitan en nuevos desarrollos.

Este trabajo parte de esa contradicción: ¿por qué las vulnerabilidades conocidas siguen ocurriendo? En contextos donde prima la entrega rápida de funcionalidad y hay poca formación en ciberseguridad, las buenas prácticas se relegan. En especial, cuando se utilizan \ACRshort{cms} como WordPress \cite{IONOS_CMS} —que concentra más del 65 \% del mercado—, el riesgo se amplifica debido a su historia de vulnerabilidades.

El objetivo de esta tesis es identificar y clasificar los antipatrones de vulnerabilidades \cite{Nafees_2019} en WordPress, analizando cómo y por qué se repiten ciertos errores, con el fin de contribuir a su prevención desde etapas tempranas del desarrollo.
