\chapter*{Motivación}
Las vulnerabilidades conocidas se registran desde finales del siglo XX en CVE. Y sea una vulnerabilidad de día cero o conocida, se trata de un resultado indeseable que genera decididamente consecuencias negativas para los usuarios. Esto es un antipatrón \cite{Brown_1998}\cite{Corry_2021}. Documentar vulnerabilidades como vulnerability antipattern (VAP) permite traducir conocimiento del dominio de la ciberseguridad\cite{CYBOK} al de la ingeniería de software\cite{SWEBoK} y dependiendo de los autores, la documentación contiene desde una plantilla estándar hasta recomendaciones en términos de acciones para las etapas de requerimientos, diseño e implementación\cite{Nafees_2019} que guían al desarrollador en la dirección de un software seguro.

\section{Objetivo}

Diseñar una aplicación web para recuperación de información sobre antipatrones de vulnerabilidades vinculados al content management system (CRM) Wordpress.

\section{Objetivos parciales}

\begin{itemize}
    \item Analizar documentación referida a antipatrón de vulnerabilidades.
    \item Diseñar template de antipatrón.
    \item Diseñar una API Rest para permitir consulta de antipatrones.
    \item Relevar los antipatrones de vulnerabilidades vinculados a CRM Wordpress.
    \item Persistir en una base de datos los antipatrones relevados.
    \item Diseñar una aplicación web que permita recuperar información de categorías de antipatrón de vulnerabilidades en particular usando la API Rest.
    \item Publicar el trabajo realizado.
    \item Escribir informe de proyecto integrador.
\end{itemize}

\section{Metodología de Trabajo}

\textbf{Lugar y herramientas}
\vspace{0.3cm}

El proyecto será llevado a cabo en la Prosecretaría de Informática de la Universidad
Nacional de Córdoba en su mayor parte y se dedicará un tiempo adicional en el domicilio de
la estudiante.
Las herramientas que incluyen tanto terminales fijos, software (gratis y de código abierto en
lo posible), servidores y equipamiento de red serán provistos por la Prosecretaría de
Informática. Los terminales cuentan con un sistema operativo Ubuntu Linux 18.04. Se
tendrá acceso a ciertos logs de servidores y flujos de datos entre dispositivos de red
mientras estos sean solicitados.

\vspace{0.3cm}
\textbf{Tiempo destinado}
\vspace{0.3cm}

Se destinará un total de 20 horas semanales distribuidas en 4 días a la semana en la Prosecretaría de Informática durante 40 semanas comenzando el 1° de Junio de 2024 y finalizando el 28 de Febrero 2025. Se destinan adicionalmente un promedio de 5 horas en la semana para tareas de investigación en fuera del horario en Prosecretaría de Informática.

\vspace{0.3cm}
\textbf{Metodología de desarrollo}
\vspace{0.3cm}

Se hará uso de una metodología iterativa incremental, donde se evaluarán los resultados de los objetivos parciales al momento y su validación, inconvenientes surgidos y próximos objetivos a alcanzar validados por el Director del Proyecto Integrador. En cada etapa final de objetivos parciales se realizará un informe de los avances.

\vspace{0.3cm}
\textbf{Supervisión y seguimiento}
\vspace{0.3cm}

Todas las etapas de desarrollo definidas serán llevadas a cabo bajo la supervisión del Director del Proyecto Integrador y del personal encargado de la Prosecretaría de Informática. Con los cuales se diagraman reuniones en conjunto para evaluar el estado del desarrollo y toma de decisiones.