\chapter{Diseño del Software}

\section{Requerimientos}

\paragraph{Introducción.}
Esta sección tiene como propósito definir los requerimientos funcionales y no funcionales para el desarrollo de una API que permitirá consultar una base de datos de antipatrones de vulnerabilidades. El sistema permitirá a los usuarios realizar consultas sobre una vulnerabilidad específica, ya sea proporcionando un CVE ID, el nombre de un plugin o tema de WordPress, o directamente del core de WordPress. Dado que cada vulnerabilidad tiene un CWE asociado, la API proporcionará información sobre el antipatrón relacionado con ese CWE con el objetivo de educar a los desarrolladores en buenas prácticas de seguridad. Para determinar el CWE, el sistema consultará la API de WP Vulnerability Database sobre la vulnerabilidad en cuestión.

\paragraph{Alcance.}

El software está diseñado para desarrolladores de WordPress que desean obtener información sobre vulnerabilidades y mejores prácticas de ciberseguridad en las diferentes etapas del SDLC. A través de diferentes endpoints y una interfaz sencilla, el sistema permitirá consultar vulnerabilidades tanto del core de WordPress como de plugins y temas, y proporcionará recomendaciones de seguridad basadas en antipatrones.


\subsection{Requerimientos funcionales}

\subsubsection{RF1: Consulta de CVE}

\textbf{Descripción:} el servicio permitirá al usuario realizar una solicitud GET al endpoint /cwe/{cwe-id} para obtener detalles sobre un antipatrón específico. La respuesta incluirá la información del antipatrón de vulnerabilidad si está disponible.

\textbf{Endpoint:} GET /cwe/{cwe-id}

\textbf{Parámetros de entrada:} 
\begin{itemize}
    \item cwe-id (obligatorio): ID del CWE, por ejemplo, 22.
\end{itemize}

\textbf{Flujo normal:}

\begin{enumerate}
    \item El usuario envía una solicitud GET al endpoint con un CWE.
    \item El sistema consulta la base de datos para buscar un documento cuyo CWE sea igual al solicitado.
    \item Se devuelve al usuario la información del antipatrón.
\end{enumerate}

\subsubsection{RF2: Consulta de versión del Kernel de Wordpress}

\textbf{Descripción:} el servicio permitirá al usuario realizar una solicitud GET al endpoint /core/{wordpress-version} para obtener detalles sobre una versión de Wordpress en específico. La respuesta incluirá la información de las vulnerabilidades de esa versión y de sus antipatrones relacionados.

\textbf{Endpoint:} GET /core/{wordpress-version}

\textbf{Parámetros de entrada:}

\begin{itemize}
    \item wordpress-version (obligatorio): versión de wordpress por la que quiere consultar. Ej.: 6.6.2
\end{itemize}


\textbf{Flujo normal:}

\begin{enumerate}
    \item El usuario envía una solicitud GET al endpoint con una versión de Wordpress.
    \item El servicio consulta la API de WP Vulnerability Database.
    \item Se obtienen todas las vulnerabilidades de esa versión y se vincula cada CWE con un antipatrón.
    \item Se devuelve al usuario la información del CVE, CWE y el antipatrón.
\end{enumerate}

\subsubsection{RF3: Consulta de plugin de Wordpress}

\textbf{Descripción:} el servicio permitirá al usuario realizar una solicitud GET al endpoint /plugins/{plugin-name}/{plugin-version} para obtener detalles sobre una versión de un plugin en específico. La respuesta incluirá la información de las vulnerabilidades de esa versión y de sus antipatrones relacionados. También podrá no especificarse la versión y obtener respuesta sobre todas las versiones de ese plugin.

\textbf{Endpoint:} GET /plugins/{plugin-name}/{plugin-version}

\textbf{Parámetros de entrada: }

\begin{itemize}
    \item plugin-name (obligatorio): nombre del plugin por el que se quiere consultar.
    \item plugin-version (opcional): versión del plugin por la que se quiere consultar.
\end{itemize}

\textbf{Flujo normal 1:}

\begin{enumerate}
    \item El usuario envía una solicitud GET al endpoint con una versión de un plugin de Wordpress.
    \item El servicio consulta la API de WP Vulnerability Database.
    \item Se obtienen todas las vulnerabilidades del plugin en esa versión y se vincula cada CWE con un antripatrón.
    \item Se devuelve al usuario la información del CVE, CWE y el antipatrón por cada vulnerabilidad.
\end{enumerate}

\textbf{Flujo normal 2:}

\begin{enumerate}
    \item El usuario envía una solicitud GET al endpoint con el nombre de un plugin de Wordpress sin especificar una versión.
    \item El servicio consulta la API de WP Vulnerability Database.
    \item Se obtienen todas las vulnerabilidades del plugin en todas sus versiones y se vincula cada CWE con un antripatrón.
    \item Se devuelve al usuario la información del CVE, CWE y el antipatrón por cada vulnerabilidad.
\end{enumerate}

\subsubsection{RF4: Consulta de tema de Wordpress}

\textbf{Descripción:} el servicio permitirá al usuario realizar una solicitud GET al endpoint /themes/{theme-name}/{theme-version} para obtener detalles sobre una versión de un tema en específico. La respuesta incluirá la información de las vulnerabilidades de esa versión y de sus antipatrones relacionados.  También podrá no especificarse la versión y obtener respuesta sobre todas las versiones de ese tema.

\textbf{Endpoint:} GET /themes/{theme-name}/{theme-version}

\textbf{Parámetros de entrada: }

\begin{itemize}
    \item theme-name (obligatorio): nombre del tema por el que se quiere consultar.
    \item theme-version (opcional): versión del tema por la que se quiere consultar.
\end{itemize}

\textbf{Flujo normal 1:}

\begin{enumerate}
    \item El usuario envía una solicitud GET al endpoint con una versión de un tema de Wordpress.
    \item El servicio consulta la API de WP Vulnerability Database.
    \item Se obtienen todas las vulnerabilidades del tema en esa versión y se vincula cada CWE con un antripatrón.
    \item Se devuelve al usuario la información del CVE, CWE y el antipatrón por cada vulnerabilidad.
\end{enumerate}

\textbf{Flujo normal 2:}

\begin{enumerate}
    \item El usuario envía una solicitud GET al endpoint con el nombre de un tema de Wordpress sin especificar una versión.
    \item El servicio consulta la API de WP Vulnerability Database.
    \item Se obtienen todas las vulnerabilidades del tema en todas sus versiones y se vincula cada CWE con un antripatrón.
    \item Se devuelve al usuario la información del CVE, CWE y el antipatrón por cada vulnerabilidad.
\end{enumerate}

\subsubsection{RF5: Manejo de errores}

El sistema deberá delegar en el framework FastAPI el manejo automático de errores estructurales, como rutas inexistentes (404) o parámetros mal formateados o ausentes (422). Esto garantizará respuestas HTTP estándar sin necesidad de implementar validaciones adicionales manuales, salvo en aquellos casos que requieran lógica de negocio específica.

\subsubsection{RF6: Base de Datos de Antipatrones}

El sistema deberá contar con una base de datos que almacene los CWE y sus correspondientes VAP.

\subsection{Requerimientos no funcionales}

\subsubsection{RNF1: Seguridad}

RNF1: El sistema debe implementar medidas de seguridad que protejan las consultas contra ataques de inyección, DDoS y otras amenazas comunes.

\subsubsection{RNF1: Disponibilidad}

RNF2: El sistema debe estar disponible 24/7 con un tiempo de inactividad permitido de no más del 1\% al mes.


\section{Diseño de la base de datos}

Durante el desarrollo del sistema de consulta de antipatrones de vulnerabilidades (VAPs), surgió la necesidad de almacenar de manera estructurada toda la información extraída del análisis de distintas vulnerabilidades en entornos WordPress. En un principio, opté por utilizar una base de datos relacional en MySQL, siguiendo los lineamientos y herramientas enseñadas durante la carrera. Este modelo, ampliamente conocido, se basa en la organización de los datos en tablas relacionadas mediante claves primarias y foráneas.

Diseñé una primera versión de la base de datos con una única tabla llamada "antipatrones", que contenía campos como el nombre del VAP, su correspondiente CWE, las etapas del SDLC en las que aparece, consecuencias, fuerzas desbalanceadas, soluciones propuestas, patrones relacionados, entre otros. Sin embargo, al intentar volcar los primeros registros, empecé a notar que esta estructura no se ajustaba adecuadamente a la naturaleza de los datos.

Muchos campos requerían almacenar múltiples valores, como por ejemplo “aka”, “fuerzas desbalanceadas”, “consecuencias”, “soluciones”, “ejemplos” o “patrones relacionados”. Para representar estos datos correctamente en un modelo relacional, habría sido necesario normalizar la base y distribuirlos en múltiples tablas relacionadas. Esta solución, aunque técnicamente correcta, aumentaba la complejidad del diseño y dificultaba el mantenimiento, especialmente considerando que cada antipatrón es una entidad independiente, sin relaciones reales con otros VAPs.

A partir de este conflicto, comencé a investigar otras opciones de almacenamiento y encontré que las bases de datos no relacionales ofrecían una alternativa mucho más adecuada para este caso. En particular, MongoDB, una base orientada a documentos, me permitió representar cada antipatrón como un documento independiente en formato JSON. Esta estructura no solo era más natural y flexible para los datos que necesitaba almacenar, sino que también facilitaba las consultas y la expansión futura de la base.

Finalmente, decidí migrar completamente la base de datos a MongoDB. Cada antipatrón fue transformado en un documento que incluye todos sus campos relevantes: nombre, cwe\_id, aka, etapa\_sdlc, fuerzas\_desbalanceadas, problema, consecuencias, solucion\_sdlc, ejemplos, ejemplos\_solucion, patrones\_relacionados, entre otros. La posibilidad de utilizar listas y anidar campos complejos me permitió mantener la estructura de cada VAP tal como estaba organizada en el informe base, sin tener que forzarla a encajar en un modelo tabular.

Subí esta base de datos a la nube utilizando MongoDB Atlas, lo cual me permitió disponer de un entorno accesible remotamente para futuras consultas a través de la API que formará parte del sistema desarrollado. Gracias a este enfoque, obtuve una solución más coherente con la estructura de la información, fácil de mantener y escalar, y alineada con las mejores prácticas actuales en el manejo de datos no estructurados.


\section{Desarrollo de la API}

\subsubsection{Elección del Lenguaje de Programación y Diseño del Proyecto}

Para el desarrollo de la API que consulta vulnerabilidades de WordPress y vincula los CWE con antipatrones, decidí utilizar Python como lenguaje de programación.
Elegí Python porque:
\begin{itemize}
    \item Ofrece una sintaxis simple y clara, ideal para avanzar rápidamente en un proyecto académico.
    \item Tiene una excelente integración con MongoDB a través de librerías como pymongo, lo que facilitó trabajar con mi base de datos NoSQL.
    \item Dispone de librerías maduras como requests para realizar solicitudes HTTP y packaging para comparar versiones de software de manera precisa, dos necesidades clave para mi proyecto.
    \item La comunidad de Python es amplia y activa, lo que significa que hay abundante documentación y recursos disponibles para resolver cualquier duda o problema que pueda surgir durante el desarrollo.
    \item La versatilidad de Python permite que, si en el futuro decidiera ampliar el proyecto, podría integrar fácilmente otras funcionalidades o servicios sin necesidad de cambiar de lenguaje.
\end{itemize}

\subsubsection{Estructura del Proyecto}

Planteé una estructura modular para que fuera ordenada y escalable:
\begin{itemize}
    \item services/: carpeta donde ubico los módulos lógicos del sistema, como wpquery.py.
    \item env/: carpeta donde se encuentra el entorno virtual de Python, que contiene todas las dependencias necesarias para el proyecto.
    \item main.py: archivo de entrada de la API.
    \item db.py: archivo que contiene la conexión a la base de datos y las funciones para interactuar con ella.
\end{itemize}

Esta organización me permite mantener el proyecto limpio y facilitar futuras mejoras.

\subsubsection{Desarrollo del módulo wpquery.py}

El archivo wpquery.py es el corazón del servicio de consulta de vulnerabilidades.
Sus principales responsabilidades son:
\begin{itemize}
    \item Realizar solicitudes HTTP a la API de WordPress Vulnerability Database usando la librería requests.
    \item Procesar las respuestas en formato JSON.
    \item Detectar para un plugin, tema o versión de core de WordPress, las vulnerabilidades aplicables a una versión específica utilizando la librería packaging para comparar versiones correctamente.
    \item Filtrar y agrupar los CWE obtenidos de las vulnerabilidades detectadas.
\end{itemize}

Dado que la API de vulnerabilidades retorna los rangos de versiones afectados mediante operadores de comparación (gt, ge, lt, le), se desarrolló una función que:
\begin{itemize}
    \item Recibe como parámetros la versión objetivo y un diccionario con los rangos de versiones afectados.
    \item Evalúa cada rango de versión utilizando la librería packaging para determinar si la versión objetivo se encuentra dentro del rango especificado.
    \item Devuelve un booleano indicando si la versión objetivo es vulnerable o no.
\end{itemize}

Esto me permitió filtrar con precisión cuáles vulnerabilidades afectaban realmente a una versión específica de un plugin, un tema o el core de WordPress.

\subsubsection{Desarrollo del módulo db.py}
El archivo db.py encapsula la lógica de acceso a la base de datos MongoDB utilizada para almacenar los antipatrones asociados a cada CWE.
Sus principales responsabilidades son:
\begin{itemize}
\item Establecer una conexión con la base de datos MongoDB utilizando la librería \texttt{pymongo}.
\item Consultar la colección de antipatrones para buscar un documento por identificador de CWE.
\item Devolver el documento correspondiente en formato de diccionario de Python.
\end{itemize}

El diseño de este módulo facilita la reutilización de la lógica de acceso a los datos desde diferentes puntos del proyecto, manteniendo el código desacoplado y ordenado.

\subsubsection{Desarrollo del módulo main.py}
El archivo main.py actúa como punto de entrada principal de la API, implementada con FastAPI.
Sus principales responsabilidades son:
\begin{itemize}
\item Definir las rutas de la API para consultar vulnerabilidades y antipatrones.
\item Integrarse con los módulos \texttt{db.py} y \texttt{wpquery.py} para obtener datos desde la base de datos y la API externa de vulnerabilidades.
\item Estructurar las respuestas, agrupando los CVE detectados por CWE y asociando el antipatrón correspondiente si está disponible.
\item Proveer endpoints específicos para consultar por el core de WordPress, plugins y temas, permitiendo además filtrar por versión.
\end{itemize}

Gracias a este módulo, es posible acceder a los resultados del análisis de vulnerabilidades de forma estructurada y coherente, facilitando su consumo por otras aplicaciones o herramientas.



\subsubsection{Tratamiento de CWE y Mapeo hacia Antipatrones}

Mientras avanzaba en el desarrollo, me di cuenta de que muchas vulnerabilidades traídas por la API estaban asociadas a CWE que yo no había trabajado directamente en mi proyecto de antipatrones por no ser parte del top de vulnerabilidades frecuentes de WordPress.

Para no perder estos casos, investigué la estructura de relaciones entre CWEs y descubrí que podían estar conectados mediante varios tipos de relaciones: ParentOf, CanFollow, CanAlsoBe, PeerOf, entre otras.

Después de analizarlo cuidadosamente, decidí trabajar solamente con relaciones ParentOf, porque:
\begin{itemize}
    \item Son relaciones de herencia directa entre vulnerabilidades, lo que asegura que tienen un vínculo fuerte y aplicable.
    \item Las otras relaciones (como CanFollow o CanAlsoBe) son condicionales o contextuales, y podrían traer confusión o falsos positivos si las tomaba en cuenta.
\end{itemize}


Finalmente, definí que solo consideraré vulnerabilidades que estén directamente asociadas a los CWE principales o bien, que tengan como padre uno de estos CWE. De esta manera, garantizo que los antipatrones ofrecidos como solución estén directamente relacionados al tipo de vulnerabilidad detectada, manteniendo la coherencia y calidad de las respuestas de la API.



