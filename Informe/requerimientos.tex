\chapter{Diseño del Software}

\section{Requerimientos}

\paragraph{Introducción.}
Esta sección tiene como propósito definir los requerimientos funcionales y no funcionales para el desarrollo de una API que permitirá consultar una base de datos de antipatrones de vulnerabilidades. El sistema permitirá a los usuarios realizar consultas sobre una vulnerabilidad específica, ya sea proporcionando un CVE ID, el nombre de un plugin o tema de WordPress, o directamente del core de WordPress. Dado que cada vulnerabilidad tiene un CWE asociado, la API proporcionará información sobre el antipatrón relacionado con ese CWE con el objetivo de educar a los desarrolladores en buenas prácticas de seguridad. Para determinar el CWE, el sistema consultará la API de WP Vulnerability Database sobre la vulnerabilidad en cuestión.

\paragraph{Alcance.}

El software está diseñado para desarrolladores de WordPress que desean obtener información sobre vulnerabilidades y mejores prácticas de ciberseguridad en las diferentes etapas del SDLC. A través de diferentes endpoints y una interfaz sencilla, el sistema permitirá consultar vulnerabilidades tanto del core de WordPress como de plugins y temas, y proporcionará recomendaciones de seguridad basadas en antipatrones.


\subsection{Requerimientos funcionales}

\subsubsection{RF1: Consulta de CVE}

\textbf{Descripción:} el servicio permitirá al usuario realizar una solicitud GET al endpoint /cve/{cve-id} para obtener detalles sobre un CVE específico. La respuesta incluirá la información de la vulnerabilidad y del antipatrón de vulnerabilidad si está disponible.

\textbf{Endpoint:} GET /cve/{cve-id}

\textbf{Parámetros de entrada:} 
\begin{itemize}
    \item cve-id (obligatorio): ID del CVE, por ejemplo, CVE-2023-1234.
\end{itemize}

\textbf{Respuestas:}

\begin{itemize}
    \item 200 OK: Devuelve el CWE y el antipatrón asociado al CVE.
    \item 400 Bad Request: Si el formato del CVE ingresado no es válido.
    \item 404 Not Found: El CVE no existe.
\end{itemize}


\textbf{Flujo normal:}

\begin{enumerate}
    \item El usuario envía una solicitud GET al endpoint con un CVE.
    \item El sistema consulta la API de WP Vulnerability Database.
    \item Se obtiene el CWE y, basado en eso, el antipatrón de la base de datos interna.
    \item Se devuelve al usuario la información del CVE, CWE y el antipatrón.
\end{enumerate}

\subsubsection{RF2: Consulta de versión del Kernel de Wordpress}

\textbf{Descripción:} el servicio permitirá al usuario realizar una solicitud GET al endpoint /core/{wordpress-version} para obtener detalles sobre una versión de Wordpress en específico. La respuesta incluirá la información de las vulnerabilidades de esa versión y de sus antipatrones relacionados.

\textbf{Endpoint:} GET /core/{wordpress-version}

\textbf{Parámetros de entrada:}

\begin{itemize}
    \item wordpress-version (obligatorio): versión de wordpress por la que quiere consultar. Ej.: 6.6.2
\end{itemize}

\textbf{Respuestas:}

\begin{itemize}
    \item 200 OK: Devuelve las vulnerabilidades de la versión con sus respectivos CWE y antipatrones.
    \item 400 Bad Request: Si el formato de la versión ingresada no respeta el el versionado semántico.
\end{itemize}

\textbf{Flujo normal:}

\begin{enumerate}
    \item El usuario envía una solicitud GET al endpoint con una versión de Wordpress.
    \item El servicio consulta la API de WP Vulnerability Database.
    \item Se obtienen todas las vulnerabilidades de esa versión y se vincula cada CWE con un antipatrón.
    \item Se devuelve al usuario la información del CVE, CWE y el antipatrón.
\end{enumerate}

\subsubsection{RF3: Consulta de plugin de Wordpress}

\textbf{Descripción:} el servicio permitirá al usuario realizar una solicitud GET al endpoint /plugins/{plugin-name}/{plugin-version} para obtener detalles sobre una versión de un plugin en específico. La respuesta incluirá la información de las vulnerabilidades de esa versión y de sus antipatrones relacionados. También podrá no especificarse la versión y obtener respuesta sobre todas las versiones de ese plugin.

\textbf{Endpoint:} GET /plugins/{plugin-name}/{plugin-version}

\textbf{Parámetros de entrada: }

\begin{itemize}
    \item plugin-name (obligatorio): nombre del plugin por el que se quiere consultar.
    \item plugin-version (opcional): versión del plugin por la que se quiere consultar.
\end{itemize}

\textbf{Respuestas:}

\begin{itemize}
    \item 200 OK: Devuelve las vulnerabilidades de la versión con sus respectivos CWE y antipatrones.
    \item 400 Bad Request: Si el formato de la versión ingresada no respeta el versionado semántico.
\end{itemize}

\textbf{Flujo normal 1:}

\begin{enumerate}
    \item El usuario envía una solicitud GET al endpoint con una versión de un plugin de Wordpress.
    \item El servicio consulta la API de WP Vulnerability Database.
    \item Se obtienen todas las vulnerabilidades del plugin en esa versión y se vincula cada CWE con un antripatrón.
    \item Se devuelve al usuario la información del CVE, CWE y el antipatrón por cada vulnerabilidad.
\end{enumerate}

\textbf{Flujo normal 2:}

\begin{enumerate}
    \item El usuario envía una solicitud GET al endpoint con el nombre de un plugin de Wordpress sin especificar una versión.
    \item El servicio consulta la API de WP Vulnerability Database.
    \item Se obtienen todas las vulnerabilidades del plugin en todas sus versiones y se vincula cada CWE con un antripatrón.
    \item Se devuelve al usuario la información del CVE, CWE y el antipatrón por cada vulnerabilidad.
\end{enumerate}

\subsubsection{RF4: Consulta de tema de Wordpress}

\textbf{Descripción:} el servicio permitirá al usuario realizar una solicitud GET al endpoint /themes/{theme-name}/{theme-version} para obtener detalles sobre una versión de un tema en específico. La respuesta incluirá la información de las vulnerabilidades de esa versión y de sus antipatrones relacionados.  También podrá no especificarse la versión y obtener respuesta sobre todas las versiones de ese tema.

\textbf{Endpoint:} GET /themes/{theme-name}/{theme-version}

\textbf{Parámetros de entrada: }

\begin{itemize}
    \item theme-name (obligatorio): nombre del tema por el que se quiere consultar.
    \item theme-version (opcional): versión del tema por la que se quiere consultar.
\end{itemize}

\textbf{Respuestas:}

\begin{itemize}
    \item 200 OK: Devuelve las vulnerabilidades de la versión con sus respectivos CWE y antipatrones.
    \item 400 Bad Request: Si el formato de la versión ingresada no respeta el versionado semántico.
\end{itemize}

\textbf{Flujo normal 1:}

\begin{enumerate}
    \item El usuario envía una solicitud GET al endpoint con una versión de un tema de Wordpress.
    \item El servicio consulta la API de WP Vulnerability Database.
    \item Se obtienen todas las vulnerabilidades del tema en esa versión y se vincula cada CWE con un antripatrón.
    \item Se devuelve al usuario la información del CVE, CWE y el antipatrón por cada vulnerabilidad.
\end{enumerate}

\textbf{Flujo normal 2:}

\begin{enumerate}
    \item El usuario envía una solicitud GET al endpoint con el nombre de un tema de Wordpress sin especificar una versión.
    \item El servicio consulta la API de WP Vulnerability Database.
    \item Se obtienen todas las vulnerabilidades del tema en todas sus versiones y se vincula cada CWE con un antripatrón.
    \item Se devuelve al usuario la información del CVE, CWE y el antipatrón por cada vulnerabilidad.
\end{enumerate}

\subsubsection{RF5: Base de Datos de Antipatrones}

El sistema deberá contar con una base de datos que almacene los CWE y sus correspondientes VAP.

\subsection{Requerimientos no funcionales}

\subsubsection{RNF1: Seguridad}

RNF1: El sistema debe implementar medidas de seguridad que protejan las consultas contra ataques de inyección, DDoS y otras amenazas comunes.

\subsubsection{RNF1: Disponibilidad}

RNF2: El sistema debe estar disponible 24/7 con un tiempo de inactividad permitido de no más del 1\% al mes.


\section{Diseño de la base de datos}

Para el diseño de la base de datos tuve en cuenta que se trataba de un caso en el que había mucho contenido rico en texto y muy variable dependiendo del antipatrón que se trate. Toda la información de un antipatrón es muy específica de una vulnerabilidad, y las recomendaciones que pueden hacerse al desarrollador también lo son.

En cuanto a los patrones de diseño que recomendé para ayudar a prevenir o generar soluciones frente a una vulnerabilidad, pueden repetirse en distintos VAPs. Sin embargo, no es lo mismo aplicar un patrón Proxy para control de acceso que usarlo para filtrar y validar solicitudes entrantes con el fin de evitar ejecución de código remoto. En ambos casos hace de intermediario, pero su aplicación concreta es diferente.

Además, cada CWE está asociado a un único antipatrón, y no hay relaciones entre antipatrones. Como consecuencia de este análisis, decidí que una única tabla de antipatrones era suficiente, ya que cada antipatrón es una entidad bien cerrada.

\subsubsection{Sobre la consulta de la API}

Esta estructura facilita las consultas desde la API, ya que el único dato necesario para recuperar un antipatrón es el cwe\_id. La búsqueda es directa, sin necesidad de joins ni relaciones adicionales, lo cual mejora el rendimiento y simplifica la implementación.


\subsubsection{Motor de base de datos}

Se utilizó MySQL como sistema de gestión de base de datos relacional por su facilidad de uso, soporte en múltiples entornos, y su integración con herramientas gráficas como MySQL Workbench. Dado que la base no requiere actualizaciones frecuentes, MySQL resulta suficiente y eficiente para el propósito de esta aplicación.



\subsubsection{Posible escalabilidad}

En caso de que en un futuro se quisiera extender la base de datos para incluir antipatrones de otros entornos además de WordPress, el diseño no requeriría grandes cambios. Lo único que variaría serían los ejemplos utilizados y las soluciones recomendadas, que en este proyecto están específicamente orientadas a WordPress (por ejemplo, enlaces a Patchstack y código PHP con funciones propias del framework).

Una posible mejora sería agregar un campo a la tabla que especifique a qué software pertenece cada antipatrón. Incluso se podría modularizar esa parte y crear una tabla separada de ejemplos y soluciones específicas por software, vinculada a la tabla principal de antipatrones.

Sin embargo, para el alcance de este proyecto, el enfoque está limitado exclusivamente a WordPress, por lo que se optó por mantener una única tabla con toda la información necesaria.