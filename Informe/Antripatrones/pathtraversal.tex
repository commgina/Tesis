\chapter{Apéndice E: Path Traversal}

\section{Antipatrón}

\subsection{\textbf{Nombre del Antipatrón:} Path Traversal}

\subsection{\textbf{También conocido como:} Directory Traversal, File Path Traversal}

\subsection{\textbf{Etapa del SDLC donde se expone:}}

Diseño, Implementación.

\subsection{\textbf{Mapping con CWE}}
CWE-22: Improper Limitation of a Pathname to a Restricted Directory ('Path Traversal')

\subsection{\textbf{Ejemplos de CVE}}
CVE-2024-0221
CVE-2024-9047

\subsection{\textbf{Ejemplo del Antipatrón:} }

\begin{itemize}
    \item https://cwe.mitre.org/data/definitions/22.html
    \item https://owasp.org/www-community/attacks/Path\_Traversal
\end{itemize}

\subsection{\textbf{Fuerzas desbalanceadas:}}

1. Entrada del usuario no restringida: La necesidad de ofrecer al usuario la opción de cargar archivos o acceder a recursos externos puede derivar en la manipulación de rutas si no se realiza una validación adecuada.
2. Acceso directo a archivos del servidor: El sistema permite acceder a archivos del servidor en función de entradas proporcionadas por el usuario, lo que puede llevar a la exposición de archivos sensibles si no se controla adecuadamente la ruta solicitada.

\subsection{\textbf{Attack Pattern}}
CAPEC-126: Path Traversal

\subsection{\textbf{Problema}}

El producto utiliza una entrada externa para construir una ruta que tiene la intención de identificar un archivo o directorio ubicado debajo de un directorio padre restringido, pero el producto no neutraliza adecuadamente los elementos especiales dentro de la ruta que pueden hacer que la ruta se resuelva en una ubicación fuera del directorio restringido.

Muchas operaciones de archivos están destinadas a realizarse dentro de un directorio restringido. Al usar elementos especiales como ".." y los separadores de "/", los atacantes pueden escapar de la ubicación restringida para acceder a archivos o directorios que se encuentran en otra parte del sistema. Uno de los elementos especiales más comunes es la secuencia "../", que en la mayoría de los sistemas operativos modernos se interpreta como el directorio padre de la ubicación actual. Esto se conoce como traversal relativo de ruta.

El traversal de ruta también abarca el uso de nombres de ruta absolutos como "/usr/local/bin" para acceder a archivos inesperados. Esto se conoce como traversal absoluto de ruta. 

\subsection{\textbf{Consecuencia}}

1. Integridad: El atacante podría ser capaz de sobrescribir o crear archivos críticos, como programas, bibliotecas o datos importantes. Si el archivo objetivo es utilizado por un mecanismo de seguridad, el atacante podría eludir ese mecanismo. Por ejemplo, agregar una nueva cuenta al final de un archivo de contraseñas podría permitir a un atacante eludir la autenticación.

2. Conficencialidad: El atacante podría ser capaz de leer el contenido de archivos inesperados y exponer datos sensibles. Si el archivo objetivo es utilizado por un mecanismo de seguridad, el atacante podría eludir ese mecanismo. Por ejemplo, al leer un archivo de contraseñas, el atacante podría realizar ataques de fuerza bruta para intentar adivinar las contraseñas y acceder a una cuenta en el sistema.

3. Disponibilidad: El atacante podría ser capaz de sobrescribir, eliminar o corromper archivos críticos inesperados, como programas, bibliotecas o datos importantes. Esto podría impedir que el producto funcione correctamente y, en el caso de mecanismos de protección como la autenticación, tiene el potencial de bloquear a los usuarios del producto.


\section{Patrón}

\subsection{Pasos de la solución en el SDLC}

\textbf{Diseño:}

\begin{itemize}
    \item Configurar correctamente el control de acceso.
    \item Hacer cumplir el principio de privilegios mínimos.
    \item Ejecutar programas con privilegios restringidos, de manera que el proceso padre no abra más vulnerabilidades. Asegúrate de que todos los directorios, archivos temporales y de memoria se ejecuten con privilegios limitados para proteger contra la ejecución remota.
    \item Validación de entradas. Asumir que las entradas del usuario son maliciosas. Utilizar un estricto control de tipos, caracteres y codificación.
    \item Comunicación proxy al host, de modo que las comunicaciones se terminen en el proxy, sanitizando las solicitudes antes de enviarlas al servidor.
    \item Ejecutar interfaces de servidor con una cuenta no-root y/o utilizar jaulas chroot u otras técnicas de configuración para restringir privilegios, incluso si un atacante obtiene acceso limitado a los comandos.
\end{itemize}

\textbf{Implementación:}

\begin{itemize}
    \item Monitoreo de la integridad del host para archivos, directorios y procesos críticos. El objetivo es estar al tanto cuando ocurra un problema de seguridad para que se inicien actividades de respuesta a incidentes y forenses.
    \item Realizar validación de entrada para todo el contenido remoto, incluyendo contenido remoto y generado por usuarios.
    \item Realizar pruebas como pen-testing y escaneo de vulnerabilidades para identificar directorios, programas e interfaces que otorgan acceso directo a ejecutables.
    \item Usar referencias indirectas en lugar de nombres de archivos reales.
    \item Utilizar los permisos posibles en el acceso a archivos al desarrollar y desplegar aplicaciones web.
    \item Validar la entrada del usuario solo aceptando datos conocidos como válidos. Asegurarse de que todo el contenido entregado al cliente esté sanitizado de acuerdo a una especificación de contenido aceptable, utilizando un enfoque de lista blanca.
\end{itemize}

\subsection{\textbf{Solución}}

\href{https://www.wpservices.com/stay-secure-protecting-your-wordpress-site-from-directory-traversal-vulnerabilities/}{WPServicses}

\href{https://patchstack.com/academy/wordpress/securing-code/arbitrary-file-deletion/#how-to-secure}{PatchStack - File Deletion}

\href{https://patchstack.com/academy/wordpress/securing-code/arbitrary-file-read/#how-to-secure}{PatchStack - FIle Read}

\href{https://patchstack.com/academy/wordpress/securing-code/arbitrary-file-deletion/}{PatchStack - File Deletion}

\textbf{Patrones relacionados}

\begin{itemize}
    \item Facade Pattern: puede proporcionar una interfaz simplificada y controlada para el acceso a recursos del sistema de archivos, permitiendo que las rutas de archivo sean verificadas y sanitizadas antes de que se realicen las operaciones de lectura o escritura. 
    \item Funcionalidades de WordPress: utilizar funciones para sanitizar la entrada como esc\_url() y esc\_url\_raw().
    \item Deshabilitar indexado de directorios: modificando las reglas de .htaccess se puede prevenir que los atacantes enumeren el contenido de los directorios  e identifiquen posibles puntos de ataque.
\end{itemize}
