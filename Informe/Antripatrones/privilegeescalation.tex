\chapter{Privilege Escalation}
\section{Antipatrón}
\subsection*{Nombre}
Privilege Escalation
\subsection*{Tambien conocido como}
Improper Privilege Management
\subsection*{Frecuentemente expuesto en la etapa del SDLC}
Arquitectura
\subsection*{Mapeo con CWE}
CWE-269
\subsection*{Ejemplos de CVE}
\begin{itemize}
    \item CVE-2020-28035
    \item CVE-2024-52442
    \item CVE-2022-1654
\end{itemize}
\subsection*{Ejemplo de antipatrón}

\begin{itemize}
    \item \href{https://cwe.mitre.org/data/definitions/269.html}{CWE}
    \item \href{https://patchstack.com/academy/wordpress/vulnerabilities/privilege-escalation/}{Patchstack}
\end{itemize}

\subsection*{Fuerzas desbalanceadas}

\begin{itemize}
    \item El requerimiento de almacenar roles o permisos en cookies sin validación en el servidor permite a un atacante modificar las cookies para acceder a funcionalidades restringidas.
    \item El requerimiento de utilizar parámetros en la URL para identificar roles o permisos permite a un atacante modificar dichos parámetros para obtener acceso administrativo o a funciones restringidas.
    \item Implementar validaciones de autorización solo en el cliente lleva a exponer vulnerabilidades de escalado horizontal o vertical, ya que la lógica de seguridad puede ser fácilmente manipulada o eludida.
    \item Implementar una API sin restringir el acceso a endpoints a usuarios que no deberían tener autorización para accederlos.
\end{itemize}

\subsection*{Attack pattern}
\begin{itemize}
    \item CAPEC-122
    \item CAPEC-233
    \item CAPEC-58
\end{itemize}
\subsection*{Problema}
El producto no asigna, modifica, rastrea o verifica correctamente los privilegios de un actor, lo que crea un ámbito de control no intencionado para ese actor.
\subsection*{Consecuencias}
Control de acceso
\begin{itemize}
    \item Obtener privilegios escalando verticalmente
    \item Escalar horizontalmente y asumir identidad
    \item Acceso a información confidencial o a información de otros usuarios permitiendo manipular recursos asignados a esa cuenta.
    \item Un atacante con privilegios elevados puede utilizar el sistema comprometido como punto de partida para lanzar ataques adicionales dentro de la red o infraestructura
\end{itemize}

\section{Patrón}
\subsection*{Solución en el SDLC}
Arquitectura
\begin{itemize}
    \item Gestionar correctamente la configuración, administración y manejo de los privilegios.
    \item Administra explícitamente las zonas de confianza dentro del software.
    \item Seguir el principio de privilegio mínimo al asignar derechos de acceso a las entidades del sistema.
\end{itemize}
\subsection*{Ejemplo de solución}
\href{https://patchstack.com/academy/wordpress/securing-code/privilege-escalation/}{Patchstack}
\subsection*{Related patterns}

\begin{itemize}
    \item Proxy: Utiliza un objeto proxy como intermediario entre el usuario y el recurso solicitado.
Este patrón puede verificar permisos y validar la autenticidad de la solicitud antes de
delegar la operación.
    \item Facade Pattern: un facade en este contexto permitiría centralizar el acceso a los datos.
    Todas las solicitudes pasarían por la Facade, que verificará los permisos antes de permitir
    el acceso al recurso solicitado.
    \item Adapter Pattern: Un Adapter podría actuar como un traductor entre lo que el usuario
    envía (por ejemplo, un identificador de un objeto que solcite) y lo que el servidor
    realmente entiende o espera.
    \item RBAC: modelo de control de acceso que restringe el acceso a recursos y acciones en un
    sistema basado en los roles asignados a los usuarios. En lugar de asignar permisos directamente a cada usuario, estos se agrupan en roles, y los usuarios heredan los permisos
    del rol que tienen asignado.
\end{itemize}




