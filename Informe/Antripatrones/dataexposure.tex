\chapter{Data Exposure}
\section{Antipatrón}
\subsection*{Nombre}
Data exposure
\subsection*{Tambien conocido como}
Exposure of Sensitive Information to an Unauthorized Actor
Information Disclosure
Information Leak
\subsection*{Frecuentemente expuesto en la etapa del SDLC}
Arquitectura
Implementación
\subsection*{Mapeo con CWE}
CWE-200
\subsection*{Ejemplos de CVE}
CVE-2024-10285
CVE-2024-2107
CVE-2013-2203 
\subsection*{Ejemplo de antipatrón}
https://cwe.mitre.org/data/definitions/200.html
\subsection*{Fuerzas desbalanceadas}
\begin{itemize}
    \item Un sistema de login con usuario y clave que avisa al usuario de que colocó una contraseña incorrecta, está revelando información sobre la existencia de ese usuario, lo que puede facilitar ataques de fuerza bruta.
    \item El código gestiona recursos que contienen intencionalmente información sensible, pero estos recursos se hacen accesibles de forma no intencionada para actores no autorizados. 
    \item La manipulación de datos sensibles de usuario sin el cifrado correcto durante su transmisión o almacenamiento, podrían ser interceptados o accesibles a personas no autorizadas. 
    \item El sistema permite un acceso uniforme a todos los usuarios sin una estructura de roles por detrás lo que puede dar acceso a datos sensibles a usuarios no autorizados.
\end{itemize}
\subsection*{Attack pattern}
CAPEC-116
\subsection*{Problema}
El producto expone información sensible a un actor que no está explícitamente autorizado para acceder a esa información. Las exposiciones de información pueden ocurrir de diversas maneras:
    \begin{itemize}
        \item El código inserta explícitamente información sensible en recursos o mensajes que son intencionalmente accesibles para actores no autorizados, pero que no deberían contener dicha información; es decir, la información debería haber sido "limpiada" o "saneada".
        \item Una debilidad o error diferente (podria ser un plugin o tema vulnerable) inserta indirectamente la información sensible en los recursos, como un error en un script web que revela la ruta completa del sistema donde se encuentra el programa.
        \item El código gestiona recursos que contienen intencionalmente información sensible, pero estos recursos se hacen accesibles por error a actores no autorizados. En este caso, la exposición de la información es resultante; es decir, una debilidad diferente permitió el acceso a la información desde el inicio.
    \end{itemize}
\subsection*{Consecuencias}
Confidencialidad:
\begin{itemize}
    \item Exposición de datos personales o privados, afectando la privacidad de los usuarios.
    \item Pérdida de secretos comerciales o propiedad intelectual, lo cual puede impactar la competitividad de una organización.
    \item Compromiso de seguridad del sistema o la red, ya que información de configuración interna puede ser usada en ataques.
    \item Exposición de registros y metadatos, que pueden ofrecer información indirecta valiosa para atacantes.
\end{itemize}
\section{Patrón}
\subsection*{Solución en el SDLC}
Arquitectura: Compartimentar el sistema para crear áreas "seguras" donde se puedan definir límites de confianza de manera clara. No permitir que los datos sensibles salgan del límite de confianza y tener siempre precaución al interactuar con un compartimento fuera del área segura.
Asegurarse de que la compartimentación adecuada esté integrada en el diseño del sistema, y que dicha compartimentación permita y refuerce la funcionalidad de separación de privilegios. Los arquitectos y diseñadores deben basarse en el principio de mínimo privilegio para decidir el momento adecuado para usar y eliminar esos privilegios.
\subsection*{Ejemplo de solución}
\subsection*{Related patterns}

\begin{itemize}
    \item RBAC
    \item Proxy Pattern: Este patrón actúa como intermediario, controlando el acceso al objeto real y permitiendo implementar autenticación, autorización y registros de actividades (logging) en el acceso a recursos sensibles. Al controlar el acceso, el Proxy Pattern ayuda a asegurar que solo los usuarios autorizados puedan interactuar con el objeto o recurso, reduciendo el riesgo de exposición de información sensible.
    \item Facade Pattern: Este patrón oculta la complejidad interna del sistema y proporciona una interfaz simplificada, lo que disminuye el riesgo de exponer detalles internos o información confidencial. Además, al centralizar el acceso, se pueden agregar verificaciones de seguridad y validación en la fachada, evitando que usuarios no autorizados accedan al subsistema o expongan datos sensibles.
\end{itemize}