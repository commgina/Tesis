\chapter{PHP Object Injection}
\section{Antipatrón}
\subsection{Nombre}
PHP Object Injection
\subsection{Tambien conocido como}
\begin{itemize}
    \item Marshaling, Unmarshaling
    \item Pickling, Unpickling (Python)
\end{itemize}
\subsection{Frecuentemente expuesto en la etapa del SDLC}
\begin{itemize}
    \item Arquitectura
    \item Implementación
\end{itemize}
\subsection{Mapeo con CWE}
CWE-502: Deserialization of Untrusted Data
\subsection{Ejemplos de CVE}
\begin{itemize}
    \item CVE-2024-13410
    \item CVE-2025-0767
    \item CVE-2025-22526
\end{itemize}

\subsection{Ejemplo de antipatrón}
\begin{itemize}
    \item \href{https://patchstack.com/academy/wordpress/vulnerabilities/php-object-injection/}{PHP Object Injection}
    \item \href{https://cwe.mitre.org/data/definitions/502.html}{CWE}
    \item \href{https://learn.snyk.io/lesson/object-injection/?ecosystem=php}{Snyk}
\end{itemize}
\subsection{Fuerzas desbalanceadas}
La inyección de objetos ocurre cuando los datos serializados provienen de la entrada del usuario y luego se deserializan de una manera que provoca un comportamiento inesperado o no deseado en la aplicación. En el peor de los casos, la inyección de objetos puede resultar en la ejecución remota de código en el servidor que realiza la deserialización.
\subsection{Attack pattern}
CAPEC-586
\subsection{Problema}
El producto deserializa datos no confiables sin garantizar la validez de los mismos.
\subsection{Consecuencias}
Integridad: los atacantes pueden llegar a morificar datos u objetos que se creia que estaban seguros de ser modificados
\section{Patrón}

\subsection{Solución en el SDLC}

\textbf{Arquitectura}
\begin{itemize}
    \item Para evitar vulnerabilidades en la deserialización, se recomienda utilizar mecanismos de firma o sellado del lenguaje de programación, como HMAC, para asegurar que los datos no han sido modificados.
    \item También se pueden definir ciertos campos como \textit{transient} para evitar que sean deserializados, asegurando que información sensible o dependiente del entorno no sea reutilizada de forma indebida.
    \item Otra medida es aplicar criptografía para proteger los datos o el código, aunque se debe tener en cuenta que si la seguridad se implementa solo en el cliente, puede ser vulnerada en caso de compromiso del mismo.
\end{itemize}

\textbf{Implementación}
\begin{itemize}
\item En lugar de deserializar directamente, crear nuevos objetos con datos validados para evitar la ejecución de código inesperado.
\item Definir explícitamente un objeto final (final object()) para evitar la deserialización cuando no sea necesaria.
\item Restringir la deserialización solo a clases permitidas mediante una lista blanca (allowlist), evitando la ejecución de gadgets maliciosos en librerías externas.
\item Evitar el uso de tipos innecesarios o cadenas de objetos que puedan ejecutarse durante la deserialización, ya que constantemente se descubren nuevas técnicas de explotación.
\end{itemize}

\subsection{Ejemplo de solución}
\href{https://patchstack.com/academy/wordpress/securing-code/php-object-injection/}{Patchstack}

\subsection{Related patterns}
\begin{itemize}
    \item Proxy Pattern: Filtra y valida los objetos antes de ser instanciados, evitando que datos no confiables lleguen al objeto real. Útil para restringir qué clases pueden ser deserializadas.
    \item Factory Pattern: Centraliza la creación de objetos, permitiendo validar los tipos antes de instanciarlos. En lugar de unserializar objetos directamente, se puede verificar la clase y sus propiedades antes de crear instancias.
\end{itemize}

