\chapter{Arbitrary File Deletion}
\section{Antipatrón}
\subsection*{Nombre}
Arbitrary File Deletion
\subsection*{Tambien conocido como}
Arbitrary File Deletion

\subsection*{Frecuentemente expuesto en la etapa del SDLC}
Arquitectura

Implementación
\subsection*{Mapeo con CWE}
CWE-22: Arbitrary File Deletion no es mas que una consecuencia de Path Traversal en la que el atacante accede sin autorización a un directorio y borra archivos.
\subsection*{Ejemplos de CVE}
\begin{itemize}
    \item CVE-2024-12035
    \item CVE-2025-1282
\end{itemize}
\subsection*{Ejemplo de antipatrón}
\begin{itemize}
    \item \href{https://patchstack.com/articles/common-plugin-vulnerabilities-how-to-fix-them/}{Patchstack}
    \item \href{https://www.php.net/unlink}{Unlink}
    \item \href{https://patchstack.com/academy/wordpress/vulnerabilities/arbitrary-file-deletion/}{Arbitraty File Deletion}
\end{itemize}

\subsection*{Fuerzas desbalanceadas}
Acceso directo a archivos del servidor: El sistema permite acceder a archivos del servidor en función de entradas proporcionadas por el usuario, lo que puede llevar a la exposición de archivos sensibles si no se controla adecuadamente la ruta solicitada.

\subsection*{Attack pattern}
CAPEC-126: Path Traversal
\subsection*{Problema}
a aplicación permite que un usuario elimine archivos en el servidor sin las restricciones adecuadas. Esto ocurre cuando el nombre o la ruta del archivo a eliminar se construyen a partir de datos proporcionados por el usuario, sin una validación o filtrado correcto.
En el contexto de WP esto ocurre con la funcion \href{https://www.php.net/unlink}{unlink} de PHP que borra archivos. El usuario provee el argumento que se le pasara a la función pudiendo ser este el nombre de un archivo o el path a cualquier archivo del sistema.
\subsection*{Consecuencias}
Disponibilidad: Un atacante podría eliminar archivos críticos del sistema, lo que podría causar fallos en la aplicación o incluso impedir que se ejecute correctamente.
Integridad: la eliminacion de un archivo critico podria generar la inestabilidad o mal funcionamiento del sistema.
\section{Patrón}
\subsection*{Solución en el SDLC}

Arquitectura
\begin{itemize}
    \item Configurar correctamente el control de acceso.
    \item Limita las rutas de acceso a directorios específicos, de modo que el sistema nunca acceda fuera de esos directorios.
    \item Almacenar los archivos fuera del directorio raíz del servidor web (por ejemplo, fuera de la carpeta public\_html en servidores Apache) para que no sean directamente accesibles mediante solicitudes HTTP.
\end{itemize}

Implementación
\begin{itemize}
    \item Usar referencias indirectas en lugar de nombres de archivos reales. Crear un mapeo desde un conjunto de valores de entrada fijos (como IDs numéricos) hacia los nombres de archivos o URLs reales, y rechaza todas las demás entradas. Por ejemplo, el ID 1 podría mapear a "inbox.txt" y el ID 2 podría mapear a "profile.txt". 
\end{itemize}

\subsection*{Ejemplo de solución}
\href{https://patchstack.com/academy/wordpress/securing-code/arbitrary-file-deletion/}{Arbitrary File Deletion - Securing code}
\subsection*{Related patterns}
\begin{itemize}
    \item Facade Pattern: puede proporcionar una interfaz simplificada y controlada para el acceso
    a recursos del sistema de archivos, permitiendo que las rutas de archivo sean verificadas
    y sanitizadas antes de que se realicen las operaciones de lectura o escritura.
    \item Proxy: El patrón Proxy actúa como un intermediario entre el cliente y el recurso real (en este caso, los archivos). Puede controlar el acceso a los archivos y asegurarse de que solo los usuarios autorizados o los archivos específicos sean accesibles.
\end{itemize}

