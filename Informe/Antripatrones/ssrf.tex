\chapter{Apéndice D: Server Side Request Forgery}
\section{Antipatrón}
\subsection{Nombre}
Server Side Request Forgery
\subsection{Tambien conocido como}
SSRF
Cross Site Port Attack
\subsection{Frecuentemente expuesto en la etapa del SDLC}
Diseño
Implementación
\subsection{Mapeo con CWE}
CWE-918
\subsection{Ejemplos de CVE}
\begin{itemize}
    \item CVE-2024-43989
    \item CVE-2024-5021
\end{itemize}
\subsection{Ejemplo de antipatrón}

\begin{itemize}
    \item \href{https://patchstack.com/academy/wordpress/vulnerabilities/server-side-request-forgery/}{Patchstack}
    \item \href{https://cwe.mitre.org/data/definitions/918.html}{CWE}
\end{itemize}

\subsection{Fuerzas desbalanceadas}

\begin{itemize}
    \item La aplicación permite a los usuarios interactuar con recursos externos, como la descarga de archivos o la obtención de información. Esta flexibilidad genera una vulnerabilidad cuando se implementa sin restricciones de validación en las URLs externas que los usuarios pueden enviar.
    \item La implementación que permite que los usuarios envíen solicitudes sin validación hace que el servidor actúe como una "ventana" hacia su propia red interna. Un atacante puede aprovechar esto para acceder a recursos internos que normalmente no están disponibles desde el exterior, como por ejemplo una base de datos. 
\end{itemize}

\subsection{Attack pattern}
CAPEC-664
\subsection{Problema}

El servidor web recibe una URL o una solicitud similar de un componente aguas arriba y recupera el contenido de esa URL, pero no garantiza suficientemente que la solicitud se esté enviando al destino esperado.
Al proporcionar URLs a hosts o puertos inesperados, los atacantes pueden hacer que parezca que el servidor está enviando la solicitud, lo que posiblemente permite eludir controles de acceso, como cortafuegos que impiden que los atacantes accedan directamente a las URLs. El servidor puede ser utilizado como un proxy para realizar escaneos de puertos en hosts de redes internas, usar otras URLs que puedan acceder a documentos en el sistema (usando file://), o utilizar otros protocolos como gopher:// o tftp://, los cuales pueden ofrecer un mayor control sobre el contenido de las solicitudes.

\subsection{Consecuencias}

Confidencialidad: lectura de datos sensibles de la aplicación

Integridad: modificar datos o realizar acciones no autorizadas dentro de la infraestructura interna.

\section{Patrón}
\subsubsection{Solución en el SDLC}
Diseño
\begin{itemize}
    \item Restringir el uso de protocolos innecesarios como file://, gopher://, o schema://. Un atacante puede usarlos para evadir las restricciones que has establecido.
    \item Aplicar el principio de privilegios minimos que establece que un usuario solo debe recibir los derechos mínimos necesarios para realizar una operación, y solo por el tiempo estrictamente necesario.

\end{itemize}
Implementación
\begin{itemize}
    \item Usa siempre mensajes de error genéricos y no verbosos. Un actor malicioso podría usar mensajes verbosos para realizar ataques a ciegas.
    \item Validar y sanitizar correctamente la entrada proporcionada por el usuario antes de pasarla a métodos sensibles como los analizadores de URL. Considera la entrada del usuario como no confiable al escribir código.
    \item Usar validación basada en una lista de permitidos (allowlist) para las direcciones IP y nombres DNS a los que tu aplicación necesita acceder. Esto previene que un atacante intente solicitar recursos no previstos.
    \item Usar un firewall para aplicaciones web (WAF) con reglas de bloqueo estrictas para detectar, bloquear y registrar cualquier carga maliciosa o entrada no deseada.
    \item Usar las funciones de seguridad que ofrece tu proveedor de la nube para mitigar vulnerabilidades comunes. Por ejemplo, AWS Cloud ofrece el método de Instance Metadata Service Version 2 (IMDSv2) que protege contra ataques SSRF y bloquea el acceso no autorizado a metadatos.
    
\end{itemize}
\subsection{Ejemplo de solución}

\begin{itemize}
    \item \href{https://patchstack.com/academy/wordpress/securing-code/server-side-request-forgery/}{PatchStack}
\end{itemize}

\subsection{Related patterns}

\textbf{Proxy Pattern} 

Permite controlar y filtrar las solicitudes realizadas desde el servidor hacia recursos externos, lo cual es clave para mitigar el riesgo de SSRF. Proporciona una capa de control que permite filtrar, validar, y registrar las solicitudes salientes.

\textbf{Funciones de Wordpress} 

Si el plugin o tema necesita obtener o realizar una solicitud a una URL externa, podemos usar las funciones integradas de WordPress dependiendo de los métodos HTTP, como: 

\begin{itemize}
    \item wp\_safe\_remote\_head
    \item wp\_safe\_remote\_get
    \item wp\_safe\_remote\_post
    \item wp\_safe\_remote\_request
\end{itemize}

Las funciones anteriores protegerán principalmente contra la vulnerabilidad SSRF y negarán el acceso a un servicio interno. Sin embargo, ten en cuenta que actualmente las funciones anteriores no son 100% seguras y existen algunos casos raros que aún permiten el acceso a servicios internos. Por favor, consulta este artículo:

\href{https://patchstack.com/articles/exploring-the-unpatched-wordpress-ssrf/}{Explorando lo desconocido: debajo de la superficie de la vulnerabilidad SSRF sin parche en WordPress} 

\textbf{Saneamiento y Validación} 

Como ocurre con la mayoría de las vulnerabilidades, un punto crítico en los ataques SSRF es el uso de datos no confiables. Siempre debe tratarse cualquier dato que provenga del lado del cliente como no confiable.

Saneando y validando los datos proporcionados por el cliente se puede avanzar mucho en la defensa contra ataques SSRF. Una validación muy intuitiva es restringir cualquier URL que contenga "localhost" o la dirección de loopback.

