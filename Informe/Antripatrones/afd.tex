\chapter{Arbitrary File Download}
\section{Antipatrón}
\subsection*{Nombre}
Arbitraty File Download
\subsection*{Tambien conocido como}
Unrestricted File Read

\subsection*{Frecuentemente expuesto en la etapa del SDLC}
Arquitectura

Implementación
\subsection*{Mapeo con CWE}
CWE-22
\subsection*{Ejemplos de CVE}
\begin{itemize}
    \item CVE-2024-52378
    \item CVE-2024-52481
\end{itemize}

\subsection*{Ejemplo de antipatrón}
\begin{itemize}
    \item \href{https://cwe.mitre.org/data/definitions/22.html}{CWE}
    \item \href{https://patchstack.com/academy/wordpress/vulnerabilities/arbitrary-file-read/}{Patchstack}
\end{itemize}
\subsection*{Fuerzas desbalanceadas}

 \begin{itemize}
     \item El sistema permite acceder a archivos del servidor en función de entradas proporcionadas por el usuario, lo que puede llevar a la exposición de archivos sensibles si no se controla adecuadamente la ruta solicitada.
     \item Permitir la descarga de archivos arbitrarios sin restricciones permitiendo que cualquier archivo dentro del servidor sea accesible por cualquier usuario.
 \end{itemize}
 
\subsection*{Attack pattern}
CAPEC-126: Path Traversal
\subsection*{Problema}
El producto permite la lectura o descarga de archivos a los que el usuario no debería tener acceso.
\subsection*{Consecuencias}
Conficencialidad: El atacante podría ser capaz de leer el contenido de archivos ines-
perados y exponer datos sensibles. Si el archivo objetivo es utilizado por un mecanismo de
seguridad, el atacante podría eludir ese mecanismo. Por ejemplo, al leer un archivo de contrase-
ñas, el atacante podría realizar ataques de fuerza bruta para intentar adivinar las contraseñas
y acceder a una cuenta en el sistema.
\section{Patrón}
\subsection*{Solución en el SDLC}

Arquitectura
\begin{itemize}
    \item Configurar correctamente el control de acceso.
    \item Limita las rutas de acceso a directorios específicos, de modo que el sistema nunca acceda fuera de esos directorios.
    \item Almacenar los archivos fuera del directorio raíz del servidor web (por ejemplo, fuera de la carpeta public\_html en servidores Apache) para que no sean directamente accesibles mediante solicitudes HTTP.
\end{itemize}

Implementación
\begin{itemize}
    \item Usar referencias indirectas en lugar de nombres de archivos reales. Crear un mapeo desde un conjunto de valores de entrada fijos (como IDs numéricos) hacia los nombres de archivos o URLs reales, y rechaza todas las demás entradas. Por ejemplo, el ID 1 podría mapear a "inbox.txt" y el ID 2 podría mapear a "profile.txt". 
\end{itemize}

\subsection*{Ejemplo de solución}
\href{https://patchstack.com/academy/wordpress/securing-code/arbitrary-file-deletion/}{PatchStack - File Deletion}
\subsection*{Related patterns}
\begin{itemize}
    \item Facade Pattern: puede proporcionar una interfaz simplificada y controlada para el acceso
    a recursos del sistema de archivos, permitiendo que las rutas de archivo sean verificadas
    y sanitizadas antes de que se realicen las operaciones de lectura o escritura.
    \item Proxy: El patrón Proxy actúa como un intermediario entre el cliente y el recurso real (en este caso, los archivos). Puede controlar el acceso a los archivos y asegurarse de que solo los usuarios autorizados o los archivos específicos sean accesibles.
\end{itemize}

