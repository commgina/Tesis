\chapter{Race Condition}
\section{Antipatrón}
\subsection*{Nombre}
Race Condition
\subsection*{Tambien conocido como}
Concurrent Execution using Shared Resource with Improper Synchronization 
\subsection*{Frecuentemente expuesto en la etapa del SDLC}
Implementación.
\subsection*{Mapeo con CWE}
CWE-362
\subsection*{Ejemplos de CVE}
\begin{itemize}
    \item CVE-2022-45809
    \item CVE-2023-6109
\end{itemize}
\subsection*{Ejemplo de antipatrón}
\begin{itemize}
    \item \href{https://cwe.mitre.org/data/definitions/362.html}{CWE}
\end{itemize}
\subsection*{Fuerzas desbalanceadas}
Los programadores suelen asumir que ciertas operaciones son muy rápidas y no pueden ser interrumpidas por otros hilos, pero esto no siempre es cierto.
   \textbf{ Ejemplo con x++}
    A nivel de código, x++ parece una operación única y atómica, pero en realidad, cuando se traduce a instrucciones de bajo nivel, se descompone en tres pasos:
    
    \begin{enumerate}
        \item Leer el valor de x de la memoria.
        \item Sumar 1 al valor de x.
        \item Escribir el nuevo valor de x en memoria.
    \end{enumerate}
    
    Si otro hilo interfiere entre estos pasos (por ejemplo, otro hilo también hace x++ al mismo tiempo), se puede generar un error de concurrencia, como una condición de carrera.
\subsection*{Attack pattern}
\begin{itemize}
    \item CAPEC-26
    \item CAPEC-29
\end{itemize}
\subsection*{Problema}
El producto contiene una secuencia de código concurrente que requiere acceso temporal y exclusivo a un recurso compartido, pero existe una ventana de tiempo en la que otro código que se ejecuta concurrentemente puede modificar dicho recurso compartido.
\subsection*{Consecuencias}
\begin{itemize}
    \item Disponibilidad: cuando una condición de carrera permite a múltiples flujos acceder a un recurso de manera simultanea puede generar estados inciertos del programa.
    \item Confidencialidad e integridad: Un atacante podría \textbf{leer o sobrescribir datos sensibles} si la condición de carrera involucra nombres de recursos predecibles y permisos débiles.
    \item Control de acceso: si la sincronización esperada esta en código de seguridad critico como por ejemplo en la autenticación de usuarios, puede tener implicaciones graves de seguridad.
\end{itemize}

\section{Patrón}

\subsection*{Solución en el SDLC}

\textbf{Implementación} 

\begin{itemize}
    \item Utilizar funciones thread-safe.
    \item Minimizar el uso de recursos compartidos para así minimizar la necesidad de sincronización necesaria.
    \item Utilizar operaciones atómicas en variables compartidas.
    \item Utilizar mutex si hay disponible.
    \item Priorizar ciertos procesos u hilos por sobre otros al momento de acceder a los recursos compartidos.
\end{itemize}

\subsection*{Ejemplo de solución}

\begin{itemize}
    \item \href{https://cwe.mitre.org/data/definitions/362.html}{CWE}
    \item \href{https://plugins.trac.wordpress.org/changeset?sfp_email=&sfph_mail=&reponame=&new=2794107%40wp-polls%2Ftrunk&old=2729999%40wp-polls%2Ftrunk&sfp_email=&sfph_mail=}{Solución usando lock}
\end{itemize}

\subsection*{Related patterns}

\begin{itemize}
    \item Proxy Pattern: puede utilizarse para controlar el acceso a un recurso compartido.
    \item Singleton: garantiza que solo una instancia de una clase exista en el sistema, lo que puede ser útil en entornos concurrentes cuando se accede a un recurso compartido.
\end{itemize}
