\chapter{Apéndice B: Cross Site Scripting CSS}

\section{Antipatrón}

\subsection{Nombre}
Cross-Site Scripting
\subsection{Tambien conocido como}
Improper Neutralization of Input During Web Page Generation ('Cross-site Scripting')
\subsection{Frecuentemente expuesto en la etapa del SDLC}
Diseño
\subsection{Mapeo con CWE}
CWE-79. 
\subsection{Ejemplos de CVE}
\begin{itemize}
    \item CVE-2024-2194
    \item CVE-2024-5645
    \item CVE-2021-24934
\end{itemize}

\subsection{Ejemplo de antipatrón}

\begin{itemize}
    \item https://patchstack.com/academy/wordpress/vulnerabilities/cross-site-scripting/
    \item https://cwe.mitre.org/data/definitions/79.html
\end{itemize}

\subsection{Fuerzas desbalanceadas}

\begin{itemize}
    \item La ausencia de un requerimiento que para implementar un mecanismo de escape de salida que asegure que todos los datos generados a partir de entradas del usuario sean escapados antes de ser presentados en la interfaz de usuario.
    \item La falta de políticas de validación de entrada que definan qué datos son aceptables y qué formatos deben tener (por ejemplo, validar tipos de datos, rangos y formatos).
    \item La falta de un sistema de filtrado de contenido que procese el HTML proporcionado por el usuario, eliminando o escapando etiquetas y atributos potencialmente peligrosos.
\end{itemize}

\subsection{Attack pattern}
\begin{itemize}
    \item CAPEC-209
    \item CAPEC-588
    \item CAPEC-591
    \item CAPEC-592
    \item CAPEC-63
    \item CAPEC-85
\end{itemize}

\subsection{Problema}

El sistema permite que los usuarios inyecten código malicioso en una página web que será ejecutado por el navegador de otros usuarios. Esto permite a los atacantes robar información sensible, como cookies, o ejecutar acciones en nombre de otros usuarios sin su conocimiento.

\subsection{Consecuencias}

\begin{itemize}
    \item Session Hijacking: Un atacante puede robar la sesión de un usuario.
    \item Defacement: Alteración del contenido visible en la web.
    \item Robo de credenciales: Los atacantes pueden capturar información sensible.
\end{itemize}

\section{Patrón}

\subsection{Solución en el SDLC}
\textbf{Diseño}

\begin{itemize}
    \item Utilizar frameworks y bibliotecas que ofrezcan mecanismos de escape automáticos para prevenir XSS.
    \item Aplicar Content Security Policy (CSP) para reducir la capacidad de ejecutar scripts maliciosos.
    \item Implementar filtros que intercepten y validen todas las entradas de los usuarios.
\end{itemize}

\textbf{Implementación}

\begin{itemize}
    \item Escapar todas las salidas dinámicas de HTML, JavaScript, CSS, PHP, etc.
    \item Verificar y filtrar todas las entradas del usuario en el servidor.
    \item Evitar la construcción dinámica de scripts, y en su lugar utilizar funciones que gestionen el contexto seguro.
\end{itemize}

\subsection{Ejemplo de solución}

\begin{itemize}
    \item \href{https://www.php.net/manual/en/function.htmlspecialchars.php}{PHP Manual}
    \item \href{https://patchstack.com/academy/wordpress/securing-code/cross-site-scripting/}{Patchstack}
    \item \href{https://patchstack.com/academy/wordpress/vulnerabilities/cross-site-scripting/}{Patchstack}
\end{itemize}

\subsection{Patrones relacionados}

\begin{itemize}

    \item \textbf{Decorator Pattern:} Se puede usar un decorador para interceptar y codificar cada comentario antes de que se almacene o se muestre en la página, previniendo que contenido malicioso inyectado pueda ejecutarse en el navegador de otros usuarios. permite aplicar dinámicamente las medidas de seguridad necesarias, como la codificación de salida, en cada lugar donde los datos del usuario se muestren en la interfaz.
    \item \textbf{Sanitizing Input Data:} Limpiar o modificar los datos que provienen de entradas externas para eliminar cualquier contenido malicioso o no deseado, como scripts o código que podría ser usado en ataques.
    La sanitización de entradas asegura que los datos ingresados cumplan con los requisitos del subsistema y de seguridad, eliminando caracteres innecesarios que pueden representar un daño potencial. Desde el navegador del usuario, los datos de entrada viajan a través de solicitudes GET, solicitudes POST y cookies, las cuales los hackers pueden editar, modificar y manipular para obtener acceso al servidor web. La sanitización de entradas actúa como un filtro para depurar los datos codificados a medida que se trasladan al servidor web. Esto se puede hacer de tres maneras:

    \begin{itemize}
        \item White-list sanitization: Permite solo caracteres y cadenas de código válidos.
        \item Black-list sanitization: Limpia la entrada eliminando caracteres no deseados, como saltos de línea, espacios adicionales, tabulaciones, \&, y etiquetas.
        \item Escaping sanitization: Rechaza solicitudes de datos no válidas y elimina las entradas para que no sean interpretadas como código.   
    \end{itemize}
    \item \textbf{Output Encoding} 
    \href{https://docs.oracle.com/en/cloud/saas/marketing/responsys-user/OutputEncodingRequiredFields.html}{Output encoding}

\end{itemize}