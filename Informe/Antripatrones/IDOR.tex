\chapter{Insecure Direct Object Reference IDOR}
\section{Antipatrón}
\subsection*{Nombre}
Insecure Direct Object Reference IDOR
\subsection*{Tambien conocido como}
\begin{itemize}
    \item Authorization Bypass Through User-Controlled Key
    \item Broken Object Level Authorization / BOLA
    \item Horizontal Authorization
\end{itemize}
\subsection*{Frecuentemente expuesto en la etapa del SDLC}
Arquitectura
\subsection*{Mapeo con CWE}
CWE-639
\subsection*{Ejemplos de CVE}
\begin{itemize}
    \item CVE-2024-10174
    \item CVE-2019-20209
\end{itemize}
\subsection*{Ejemplo de antipatrón}
\begin{itemize}
    \item \href{https://cwe.mitre.org/data/definitions/639.html}{CWE}
    \item \href{https://portswigger.net/web-security/access-control/idor}{Portswigger}
\end{itemize}
\subsection*{Fuerzas desbalanceadas}
\begin{itemize}
    \item El sistema implementa mecanismos de autorización pero no previene que un usuario acceda a información de otro permitiendo la modificación de identificadores claves de los datos. 
    \item Los identificadores internos del sistema (como IDs de base de datos, números de factura, etc.) son enviados o expuestos directamente en las URLs o como parámetros en los formularios. Los usuarios pueden modificar estos identificadores en las URLs o solicitudes para acceder a recursos de otros usuarios o recursos a los que no deberían tener acceso.
    \item Se utilizan identificadores para acceder a recursos pero estos son predecibles o secuenciales (por ejemplo, orderID=1, orderID=2, orderID=3, etc.). 
\end{itemize}
\subsection*{Attack pattern}
\begin{itemize}
    \item CAPEC-1
    \item CAPEC-77
\end{itemize}

\subsection*{Problema}
Insecure Direct Object Reference (IDOR) es una vulnerabilidad que surge cuando los atacantes pueden acceder o modificar objetos manipulando identificadores utilizados en las URL o parámetros de una aplicación web. Esto ocurre debido a la ausencia de verificaciones de control de acceso, que no logran comprobar si un usuario debería tener permitido acceder a datos específicos.

\subsection*{Consecuencias}
Control de acceso
\begin{itemize}
    \item  La escalación vertical de privilegios es posible si el dato o parámetro controlado por el usuario es, en realidad, un indicador que señala el estado de administrador, lo que permite al atacante obtener acceso con privilegios administrativos.
    \item La escalación horizontal de privilegios es posible cuando un usuario puede ver o modificar la información de otro usuario.
\end{itemize}


\section{Patrón}
\subsection*{Solución en el SDLC}
Arquitectura
\begin{itemize}
    \item Para cada acceso a datos, asegurarse de que el usuario tenga privilegios suficientes para acceder al registro solicitado.
    \item Utilizar identificadores de objetos complejos
    \item Utilizar frameworks de control de acceso
\end{itemize}
\subsection*{Ejemplo de solución}
\begin{itemize}
    \item \href{https://cheatsheetseries.owasp.org/cheatsheets/Insecure_Direct_Object_Reference_Prevention_Cheat_Sheet.html}{OWAP Cheatseet}
    \item \href{https://blog.hackmetrix.com/insecure-direct-object-reference/}{Hackmetrix}
\end{itemize}

\subsection*{Related patterns}
\begin{itemize}
    \item Proxy: Utiliza un objeto proxy como intermediario entre el usuario y el recurso solicitado. Este patrón puede verificar permisos y validar la autenticidad de la solicitud antes de delegar la operación.
    \item Facade Pattern: un facade en este contexto permitiría centralizar el acceso a los datos. Todas las solicitudes pasarían por la Facade, que verificará los permisos antes de permitir el acceso al recurso solicitado.
    \item Adapter Pattern: Un Adapter podría actuar como un traductor entre lo que el usuario envía (por ejemplo, un identificador de un objeto que solciite) y lo que el servidor realmente entiende o espera.
    \item RBAC: modelo de control de acceso que restringe el acceso a recursos y acciones en un sistema basado en los roles asignados a los usuarios. En lugar de asignar permisos directamente a cada usuario, estos se agrupan en roles, y los usuarios heredan los permisos del rol que tienen asignado.
\end{itemize}