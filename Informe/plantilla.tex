
\textbf{Antipatrón (Problema)}

    \begin{enumerate}
        \item \textbf{Nombre del Antipatrón:} Describe el nombre del antipatrón que representa el problema en el código o sistema.
        \item \textbf{También conocido como:} Otros nombres o sinónimos utilizados para describir este antipatrón.
        \item \textbf{Etapa del \ACRshort{sdcl} donde se expone:} Fase del ciclo de vida de desarrollo de software en la que la vulnerabilidad se hace visible. 
        \item \textbf{Mapping con \ACRshort{cwe}:} Relación del antipatrón con una entrada de CWE para identificar la vulnerabilidad en términos de ID y nombre.
        \item \textbf{CVE Asociados a cada componente de la arquitectura:} Algunos CVE-ID relacionados a cada componente de Wordress (core, tema y plugin).
        \item \textbf{Ejemplo del antipatrón:} Enlaces a ejemplos de explotación de códigos vulnerables. 
        \item \textbf{Fuerzas desbalanceadas:} Decisiones de implementación que intentan cumplir con los requerimientos pero que crean la vulnerabilidad. Son a modo de ejemplo ya que los requerimientos pueden variar dependiendo la aplicación.
        \item \textbf{Attack Pattern:} Identificación de CAPEC-ID.
        \item \textbf{Problema:} Breve descripción del problema basado en la información del CWE.
        \item \textbf{Consecuencia:} Consecuencias de la explotación. Como afecta a la integridad, confidencialidad y disponibilidad. Si afecta a los tres componentes o a alguno de ellos.
    \end{enumerate}

\textbf{Patrón}

    \begin{enumerate}
        \item \textbf{Soluciones en el SDLC:}  Acciones a tener en cuenta para mitigar el problema. No siempre aparecen acciones para las tres etapas del SDLC. Estas acciones surgen de las recomendaciones de el CWE y CAPEC.
        \item \textbf{Solución del antipatrón:} Enlaces a ejemplos de solución. A veces esta solución no se encuentra, por lo tanto se puede consultar a ChatGPT sobre como solucionar el código vulnerable
        \item \textbf{Patrones relacionados:} patrones de desarrollo de software que pueden ayudar a abordar el antipatrón, con soluciones específicas para tecnologías o lenguajes de programación.
    \end{enumerate}