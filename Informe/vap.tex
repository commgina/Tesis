
\chapter{Antipatrón de Vulnerabilidades}    

Antes de saltar directamente al tema Wordpress, es de suma importancia revisar los conceptos de \gls{patrón}, \gls{antipatrón} y \gls{vulnerabilidad} presentes en el glosario para asi poder entender el porque de la necesidad de relacionarlos en una plantilla y posteriormente, en una base de datos de \ACRshort{vap}.

\vspace{0.3cm}
Un \gls{antipatrón} es algo que generalmente se considera negativo en el contexto del desarrollo de software. Se refiere a una \textit{práctica, diseño o solución a un problema que, aunque pueda parecer válido o común, en realidad lleva a resultados negativos}, como código difícil de mantener, bajo rendimiento o vulnerabilidades de seguridad.
Entonces, en términos generales, un anti-patrón es algo que se debe evitar en el desarrollo de software. Reconocer y comprender los anti-patrones puede ayudar a los desarrolladores a mejorar la calidad, la eficiencia y la seguridad del software que producen, ya que les permite evitar soluciones que pueden llevar a problemas conocidos o introducir nuevas vulnerabilidades.

\section{Perspectiva de los antipatrones para desarrolladores de Software}

Las \gls{vulnerabilidad}es de software son debilidades en la arquitectura, el diseño o el código de un sistema que pueden causar violaciones de la política de seguridad del sistema. Su existencia es la causa principal de los ataques a los sistemas de software. Aunque la ciberseguridad es un atributo de calidad fundamental, a menudo se considera como algo secundario en el desarrollo de software
Al tener otras preocupaciones, los desarrolladores piensan que la responsabilidad de la seguridad es de alguien más como los pentesters o hackers éticos. Los APV proporcionan una solución a esto en tres pasos:

\begin{enumerate}
    \item Identificar prácticas de ingeniería de software deficientes que resultaron en una vulnerabilidad.
    \item Mostrarle al desarrollador como mitigar la vulnerabilidad
    \item Motivar a adoptar mejores prácticas de seguridad
\end{enumerate}
Un Anti-Patrón de Vulnerabilidad identifica un problema (es decir, una práctica deficiente que causa negativamente una falla de seguridad) y una solución (es decir, un conjunto de acciones de refactorización que se pueden llevar a cabo para mitigar o detener las fallas).

\section{Objetivo del antipatrón de vulnerabilidades}
El objetivo es encapsular el conocimiento de vulnerabilidades que está presente en bases de datos como CVE o CAPEC, y presentarle este conocimiento al desarrollador usando un enfoque basado en patrones (al que está acostumbrado).
Disminuir la brecha de conocimiento entre expertos en ciberseguridad y desarrolladores de software.


\section{Relación entre patrón y VAP}
Para mitigar cada antipatrón, se propone un patrón correspondiente, que incluye lo siguiente: contexto, problema, fuerzas y mitigación. Juntos, el anti-patrón y el patrón pueden permitir a los desarrolladores reconocer las causas raíz de una vulnerabilidad y ayudarles a entender cómo esta vulnerabilidad podría ser explotada por hackers malintencionados.


Para presentar esta relación se utiliza una plantilla, la cual contendrá toda la información necesaria para que el desarrollador comprenda cual es el origen, el modo de explotación y las maneras de mitigar las diferentes vulnerabilidades que posee su software.